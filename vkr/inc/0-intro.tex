\anonsection{Введение}

Облачные услуги --- это способ предоставления, потребления и управления технологией.
Данный тип услуг выводит гибкость и эффективность на новый уровень, путем эволюции способов управления, таких как непрерывность, безопасность, резервирование и самообслуживание, которые соединяют физическую и виртуальную среду.

Для эффективной работы облачной инфраструктуры требуется эффективная структура и организация.
Небольшая команда из специалистов и бизнес-пользователей может создать обоснованный план и организовать свою работу в подобной инфраструктуре.
Данная выделенная группа может намного эффективнее построить и управлять нестандартной облачной инфраструктурой, чем если компании будут просто продолжать добавлять дополнительные сервера и сервисы для поддержки центра обработки данных (\hyperlink{dc}{ЦОД}).

Развитие информационного мира движется в сторону повсеместного распространения облачных вычислений, их технологий и сервисов.
Очевидные преимущества данного подхода \cite{telecom-world}:
\begin{itemize}
  \item снижение затрат --- отсутствие необходимости покупки собственного оборудования, программного обеспечения (\hyperlink{soft}{ПО}), работы системного инженера;
  \item удаленный доступ --- возможность доступа к данным облака из любой точки мира, где есть доступ в глобальную сеть Интернет;
  \item отказоустойчивость и масштабируемость --- изменение необходимых ресурсов в зависимости от потребностей проекта, техническое обслуживание оборудования лежит на плечах облачного провайдера.
\end{itemize}

В связи с этим можно сделать вывод, что основные недостатки облачных вычислений сводятся к информационной безопасности.
Такого мнения придерживаются многие крупные информационные компании, что в некоторой степени препятствует более стремительному развитию рынка облачных сервисов.

\clearpage
