\begingroup
\renewcommand{\section}[2]{\anonsection{Библиографический список}}
\begin{thebibliography}{00}

\bibitem{empirical-analysis}
    P. Mayer and A. Bauer, 
    <<An empirical analysis of the utilization of multiple programming languages in open source projects>>, 
    in Proceedings of the 19th International Conference on Evaluation and Assessment in Software Engineering, 
    Nanjing, China, 2015.

\bibitem{professional-developers}
    Mayer, P., Kirsch, M. \& Le, M.A. On multi-language software development,
    cross-language links and accompanying tools: 
    a survey of professional software developers. 
    J Softw Eng Res Dev 5, 1 (2017). \url{https://doi.org/10.1186/s40411-017-0035-z}


\bibitem{static-program-analysis}
    Static Program Analysis
    Anders Møller and Michael I. Schwartzbach
    Department of Computer Science, Aarhus University. May 2023.
    \url{https://cs.au.dk/~amoeller/spa/}

\bibitem{linter}
    Johnson, Stephen C. (25 October 1978).
    "Lint, a C Program Checker". 
    Comp. Sci. Tech. Rep. Bell Labs: 78–1273.
    CiteSeerX 10.1.1.56.1841. 

\bibitem{qt-moc}
    \url{https://doc.qt.io/qt-6/metaobjects.html}

\bibitem{react}
    \url{https://react.dev/}

\bibitem{aid-developers}
    Pfeiffer, RH., Wąsowski, A. (2012).
    Cross-Language Support Mechanisms Significantly Aid Software Development.
    In: France, R.B., Kazmeier, J., Breu, R., Atkinson, C. (eds)
    Model Driven Engineering Languages and Systems. MODELS 2012. 
    Lecture Notes in Computer Science, vol 7590. Springer, Berlin, Heidelberg.
    pp 168-184
    \url{https://doi.org/10.1007/978-3-642-33666-9_12}
    
\bibitem{pragmatic-evidence}
    S. Latif, Z. Mushtaq, G. Rasool, F. Rustam, N. Aslam,
    and I. Ashraf, ‘Pragmatic evidence of cross-language link detection:
    A systematic literature review’, Journal of Systems and Software,
    vol. 206, p. 111, 2023.

\bibitem{island-grammars}
    Moonen, Leon. (2001). Generating Robust Parsers using Island Grammars. 
    
\bibitem{external-dependencies}
    J. M. Fernandes, G. H. Travassos, V. Lenarduzzi, and X. Li, 
    Quality of Information and Communications Technology: 
    16th International Conference, QUATIC 2023,
    Aveiro, Portugal, September 11--13, 2023,
    Proceedings. Springer Nature Switzerland, 2023.

\bibitem{pangea}
    A. Caracciolo, A. Chis, B. Spasojevic and
    M. Lungu, "Pangea: A Workbench for Statically Analyzing Multi-language Software Corpora,"
    2014 IEEE 14th International Working Conference on Source Code Analysis
    and Manipulation, Victoria, BC, Canada, 2014,
    pp. 71-76, doi: 10.1109/SCAM.2014.39.

\bibitem{moose}
    O. Nierstrasz, S. Ducasse, and T. Gîrba, “The story of Moose: an
    agile reengineering environment,” in Proceedings of the European
    Software Engineering Conference (ESEC/FSE’05). New York, NY,
    USA: ACM Press, Sep. 2005, pp. 1–10, invited paper.

\bibitem{famix}
    S. Demeyer, S. Tichelaar, and S. Ducasse, “FAMIX 2.1 — The FAMOOS
    Information Exchange Model,” University of Bern, Tech. Rep., 2001

\bibitem{ecma262}
    \url{https://tc39.es/ecma262/}

\bibitem{JNI}
    \url{https://docs.oracle.com/en/java/javase/11/docs/specs/jni/intro.html#java-native-interface-overview}

\bibitem{MLSA}
    Anne Marie Bogar, Damian M. Lyons, David Baird,
    "Lightweight Call-Graph Construction for Multilingual Software Analysis," 2018.

\bibitem{SNDGA}
    Kargar, M., Isazadeh, A., Izadkhah, H.
    Improving the modularization quality of heterogeneous multi-programming software systems by unifying structural and semantic concepts.
    J Supercomput 76, 87–121 (2020). \url{https://doi.org/10.1007/s11227-019-02995-3}

\bibitem{polycall}
    Zheng, W., \& Hua, B. Effective Call Graph Construction for Multilingual Programs. 2023.

\bibitem{wasm}
    \url{https://webassembly.org/}

\bibitem{eCST}
    G. Rakić and Z. Budimac, ‘Introducing Enriched Concrete Syntax Trees’, arXiv [cs.SE]. 2013.

\bibitem{rust}
    \url{https://doc.rust-lang.org/1.8.0/book/references-and-borrowing.html}

\bibitem{property-graph}
    \url{https://github.com/opencypher/openCypher/blob/master/docs/property-graph-model.adoc}

\bibitem{RDF}
    \url{https://www.w3.org/TR/PR-rdf-syntax/Overview.html}

\bibitem{algorithm-W}
    Milner, Robin (1978), "A Theory of Type Polymorphism in Programming",\
    Journal of Computer and System Sciences, 17 (3): 348–375,
    doi:10.1016/0022-0000(78)90014-4, hdl:20.500.11820/d16745d7-f113-44f0-a7a3-687c2b709f66

\bibitem{scope-graphs}
    Pierre Néron, Andrew P. Tolmach, Eelco Visser, Guido Wachsmuth.
    A Theory of Name Resolution. In Jan Vitek, editor, Programming Languages and Systems
    - 24th European Symposium on Programming, ESOP 2015, Held as Part of the European Joint Conferences 
    on Theory and Practice of Software, ETAPS 2015, London, UK, April 11-18, 2015. Proceedings. 
    Volume 9032 of Lecture Notes in Computer Science, pages 205-231, Springer, 2015. [doi]

\bibitem{scope-graphs-static-analysis}
    Hendrik van Antwerpen, Pierre Néron,
     Andrew Tolmach, Eelco Visser, and Guido Wachsmuth.
      A Constraint Language for Static Semantic Analysis based on Scope Graphs with Proofs.
       Technical Report TUD-SERG-2015-012, Delft University of Technology.
    
\bibitem{scope-graphs-typed}
    A Theory of Name Resolution. Pierre Neron, Andrew P. Tolmach,
     Eelco Visser, Guido Wachsmuth. In Jan Vitek (editor),
      Programming Languages and Systems - 24th European Symposium on Programming,
       ESOP 2015, Held as Part of the European Joint Conferences on Theory and Practice of Software,
        ETAPS 2015, London, UK, April 11-18, 2015, Proceedings. 2015

\bibitem{LSP-spec}
    LSP Protocol [Электронный ресурс] //
    \url{https://microsoft.github.io/language-server-protocol/specifications/lsp/3.17/specification}
    (Дата обращения: 10.04.2024)

\end{thebibliography}
\endgroup

\clearpage
