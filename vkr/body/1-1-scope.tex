\section{Исследование существующих технологий и языков программирования в контексте
областей разработки программного обеспечения}

\subsection{Современная индустрия разработки программного обеспечения}

Современные программные проекты, в отличие от многих программных проектов прошлого,
гораздо чаще состоят из набора разных (порой разительно) технологических
решений, предназначенных для решения определенного круга задач. Согласно \cite{empirical-analysis},
в одном программном проекте в среднем задействовано 5 языков программирования, один из
которых является <<основным>>, о остальные -- специализированными языками предметной области (\hyperlink{DSL}{DSL}).
В заключении статьи авторы признают популярность мультиязыковых программных проектов
и оценивают важность наличия соответствующих инструментальных средств для работы с
такого рода проектами.

Также, нередко использование нескольких языков и в индустрии разработки. Так, авторы \cite{professional-developers}
провели исследование популярности мультиязыковых проектов
в результате опроса 139 профессиональных разработчиков из разных сфер. 
Результаты опроса показали, что опрашиваемые имели дело с 7 различными языками, в среднем.
При этом, в работу было вовлечено в среднем 3 пары связанных языков в контексте одного проекта.
Более 90\% опрашиваемых также сообщали о проблемах согласованности между языками,
встречаемых при разработке в такой мультиязыковой среде. 

Таким образом, в современной разработке программного обеспечения нередко использование
нескольких языков вне зависимости от объемов проекта или вовлекаемой предметной области.
Ситуация становится сложнее со временем, так как создание новых технологий разработки
часто влечет за собой формирование определенной нотации или языка для управления или конфигурации.
Например, это может касаться таких повсеместных технологий как \hyperlink{СУБД}{СУБД}, система сборки,
сервер приложений или скрипты развертывания.

Для наглядности, можно привести следующие языки, нередко фигурирующие в составе современных программных проектов:
\begin{itemize}    
    \item язык разметки HTML в составе проекта, использующего ASP фреймворк,
    \item язык скриптов командной строки в составе проекта, использующего язык C,
    \item язык запросов SQL в составе проекта, использующего Python и фреймворк Flask,
    \item язык препроцессора в составе файла исходного кода, реализованного на C++
\end{itemize}
Заключительный пункт списка примеров приведен для того, чтобы показать характер
связи различных технологий — разным языкам необязательно даже находится в раздельных
файлах или модулях, нередки случаи полноценного переплетения различных синтаксисов и
семантик.

\subsection{Актуальность мультиязыкового статического анализа}



\clearpage