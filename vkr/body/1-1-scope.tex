\section{Исследование существующих технологий и языков программирования в контексте
областей разработки программного обеспечения}

\subsection{Современная индустрия разработки программного обеспечения}

Современные программные проекты, в отличие от многих программных проектов прошлого,
гораздо чаще состоят из набора разных (порой разительно) технологических
решений, предназначенных для решения определенного круга задач. Согласно \cite{empirical-analysis},
в одном программном проекте в среднем задействовано 5 языков программирования, один из
которых является <<основным>>, о остальные -- специализированными языками предметной области (\hyperlink{DSL}{DSL}).
В заключении статьи авторы признают популярность мультиязыковых программных проектов
и оценивают важность наличия соответствующих инструментальных средств для работы с
такого рода проектами.

Также, нередко использование нескольких языков и в индустрии разработки. Так, авторы \cite{professional-developers}
провели исследование популярности мультиязыковых проектов
в результате опроса 139 профессиональных разработчиков из разных сфер. 
Результаты опроса показали, что опрашиваемые имели дело с 7 различными языками, в среднем.
При этом, в работу было вовлечено в среднем 3 пары связанных языков в контексте одного проекта.
Более 90\% опрашиваемых также сообщали о проблемах согласованности между языками,
встречаемых при разработке в такой мультиязыковой среде. 

Таким образом, в современной разработке программного обеспечения нередко использование
нескольких языков вне зависимости от объемов проекта или вовлекаемой предметной области.
Ситуация становится сложнее со временем, так как создание новых технологий разработки
часто влечет за собой формирование определенной нотации или языка для управления или конфигурации.
Например, это может касаться таких повсеместных технологий как \hyperlink{СУБД}{СУБД}, система сборки,
сервер приложений или скрипты развертывания.

Для наглядности, можно привести следующие языки, нередко фигурирующие в составе современных программных проектов:
\begin{itemize}    
    \item язык разметки HTML в составе проекта, использующего ASP фреймворк,
    \item язык скриптов командной строки в составе проекта, использующего язык C,
    \item язык запросов SQL в составе проекта, использующего Python и фреймворк Flask,
    \item язык препроцессора в составе файла исходного кода, реализованного на C++.
\end{itemize}
Заключительный пункт списка примеров приведен для того, чтобы показать характер
связи различных технологий —- разным языкам необязательно даже находится в раздельных
файлах или модулях, нередки случаи полноценного переплетения различных синтаксисов и
семантик.

\subsection{Актуальность мультиязыкового статического анализа}

Итак, мультиязыковые программные проекты нередки. Следовательно, имеет смысл использования
различных техник работы с исходным кодом таких проектов, поддерживающих процесс разработки.
Одной из таких техник, обеспечивающей разные сценарии использования, является статический анализ
исходного кода. 

Стоит уточнить что имеется под определением <<статический анализ кода>>. Статический
анализ это прежде всего набор различных техник по извлечению информации о программе без
явного её запуска \cite{static-program-analysis}. Таким образом, статический анализ может
быть полезен в сценариях, которые не предполагают явного запуска программы -- в сущности во
всех сценариях процесса разработки ПО исключая этапы тестирования и запуска.

Возможные сценарии использования статического анализа кода включают (но не ограничиваются):
\begin{itemize}
    \item оптимизацию программ,
    \item выявление потенциальных уязвимостей,
    \item доказательство сохранения определенных инвариантов,
    \item сбор определенной статистики,
    \item выявление <<пахнущих>> фрагментов кода,
    \item помощь разработчику во время кодирования,
    \item автоматический рефакторинг кода.
\end{itemize}

Так как сценарии использования статического анализа настолько разнообразны, в рамках данной работы
решено было сосредоточится на аспектах разработки, которые помогают в процессе кодирования и поддержки проекта.
В качестве прикладной реализации таких аспектов выступают различные инструментальные средства.
К таким средствам, к примеру, относятся:
\begin{itemize}
    \item интегрированные средства разработки (\hyperlink{IDE}{IDE}),
    \item линтеры (собирательное название инструментов, первым из которых был <<Lint>> \cite{linter}),
    \item инструменты автоматического рефакторинга,
    \item инструменты сбора статистики,
    \item различные кодогенераторы и фреймворки \cite{qt-moc}\cite{react}.
\end{itemize}

Необходимость в таких средствах присутствует и она достаточно высока. Так, согласно
исследованию \cite{aid-developers}, использование средств поддержки разработчика
(в данной работе это механизмы анализа и навигации по межъязыковым связям) позволяет
улучшить как скорость разработки ПО, так и уменьшить количество совершаемых ошибок. Стоит заметить,
что несмотря на количество времени, прошедшее с момента проведения исследования, 
принципы разработки ПО в данной предметной области (веб-разработка) не изменились и большинство
программных проектов веб-приложений состоят как минимум из двух языков. 
Обычно это разделение проводится по принципу фронтенд и бекенд.

Также, в последние годы тема исследования межъязыковых зависимостей для обеспечения инструментального
анализа начала освещаться более подробно. Так, в статье \cite{pragmatic-evidence} рассматриваются
76 исследований ориентированных на выявление межъязыковых связей в различных предметных областях.
Авторы признают важность систематического обзора данной темы как в контексте разработки и поддержки
бизнес решений, так и в отношении академических исследований. Соответственно, существует
практическая проблема сопровождения мультиязыковых систем, которая присутствует как в промышленном ПО, так
и в академических разработках.

\clearpage