\section{Анализ состояния работ в области анализа мультиязыковых исходных текстов программ}

\subsection{Теоретические исследования в области мультиязыкового анализа}

\subsubsection{a}

\subsubsection{Универсальное решение на основе CG}

\subsubsection{Обзорное исследование работ по мультиязыковому анализу}

% TODO: Использовать это как доказательство
% Theory
% Genetic algorithm file:///home/uber/dev/mag/crosslingual-analysis/sources/Improving%20the%20modularization%20quality%20of%20heterogeneous.pdf
% Call graph MLSA: https://arxiv.org/pdf/1808.01213.pdf https://arxiv.org/pdf/1808.01210.pdf
% Grab something about object models from here
% BIG Multilingual Source Code Analysis: A Systematic Literature Review file:///home/uber/dev/mag/crosslingual-analysis/sources/07953501.pdf
% Lexical approach: /home/uber/dev/mag/crosslingual-analysis/sources/On_the_Impact_of_Interlanguage_Dependencies_in_Multilanguage_Systems_Empirical_Case_Study_on_Java_Native_Interface_Applications_JNI.pdf


\subsection{Практические решения в области мультиязыкового анализа}

\subsubsection{java}

\subsubsection{Mulang}

\subsubsection{Pangea}
% Practice
% Opensource
%? Static Code Analysis of Multilanguage Software Systems: https://arxiv.org/pdf/1906.00815.pdf
% Mulang https://mumuki.github.io/mulang/
% Pangea /home/uber/dev/mag/crosslingual-analysis/sources/Pangea_A_Workbench_for_Statically_Analyzing_Multi-language_Software_Corpora.pdf
% file:///home/uber/dev/mag/crosslingual-analysis/reports/sem2/%D0%9E%D1%80%D0%BB%D0%BE%D0%B2%D1%81%D0%BA%D0%B8%D0%B9%D0%9C%D0%AE_%D0%9E%D1%82%D1%87%D0%B5%D1%82_%D0%9F%D1%80%D0%B0%D0%BA%D1%82%D0%B8%D0%BA%D0%B0.pdf
% 2.1

\clearpage